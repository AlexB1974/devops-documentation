 HEAD
\documentclass{article}
\usepackage[utf8]{inputenc}
\usepackage{geometry}
\usepackage{hyperref}
\usepackage{fancyhdr}

\geometry{margin=1in}
\pagestyle{fancy}
\fancyfoot[C]{\href{https://github.com/AlexB1974}{GitHub: AlexB1974} \quad | \quad Author: Alex Brix \quad | \quad Stockholm, Sweden}

\begin{document}

% Humorous header
\begin{center}
\texttt{\#!/bin/sh} \\
\texttt{echo This is a vodka-bottle-documentation, sorry, no automation at this time, :-/} \\
\texttt{exit 1}
\end{center}

\vspace{1cm}

\title{DevOps Infrastructure Documentation}
\author{Alex Brix}
\date{September 14, 2025}

\maketitle

\section{Level 1 – Web Console}
\begin{itemize}
  \item Login to AWS Web Console
  \item Launch EC2 instance using public AMI
  \item SSH access
  \item Manual configuration
  \item Manual deployment from Git
\end{itemize}

\section{Level 2 – AWS CLI}
\begin{itemize}
  \item Use \texttt{aws ec2 run-instances}
  \item Transition from GUI to CLI
  \item Faster and repeatable flow
\end{itemize}

\section{Level 3 – Infrastructure as Code}

\subsection{Iteration 1 – Basic Terraform}
\begin{verbatim}
terraform init
terraform apply -auto-approve
\end{verbatim}

\subsection{Iteration 2 – Configuration with user-data}
\begin{verbatim}
#!/bin/bash
sudo yum update -y
sudo yum install httpd -y
systemctl start httpd
systemctl enable httpd
\end{verbatim}

\subsection{Iteration 3 – Auto-deploy with user-data}
Destroy everything $\rightarrow$ Add auto-deploy $\rightarrow$ Reconstruct $\rightarrow$ Verify

\subsection{Iteration 4 – Ansible Integration}
\begin{verbatim}
ansible-playbook -i inventory webserver.yml
\end{verbatim}

\subsection{Iteration 5 – Git-based Source}
The infrastructure code is hosted on GitHub at: \href{https://github.com/AlexB1974/infra}{github.com/AlexB1974/infra}

\begin{itemize}
  \item Terraform modules for EC2, VPC, and S3
  \item Ansible playbooks for web server configuration
  \item CI/CD pipeline definitions using CodePipeline and CodeBuild
  \item Verification scripts and buildspec files
\end{itemize}

To deploy the infrastructure:
\begin{verbatim}
git clone https://github.com/AlexB1974/infra.git
cd infra
terraform apply -auto-approve
\end{verbatim}

\subsection{Iteration 6 – Shared Backend}
\begin{verbatim}
terraform {
  backend "s3" {
    bucket         = "alexbrix-terraform-state"
    key            = "infra/terraform.tfstate"
    region         = "eu-north-1"
    dynamodb_table = "terraform-lock-alexbrix"
  }
}
\end{verbatim}

\subsection{Iteration 7 – Optimization}
\begin{verbatim}
terraform plan > plan.log
grep "No changes" plan.log && echo "Everything is fine"
\end{verbatim}

\subsection{Iteration 8 – Verification Script}
\begin{verbatim}
#!/bin/bash
terraform plan > verify.log
cat verify.log | grep "No changes"
\end{verbatim}

\section{Level 4 – CI/CD and Security}

\subsection{Iteration 1 – CodePipeline + CloudFormation}
\begin{verbatim}
aws cloudformation create-stack \
  --stack-name InfraStack \
  --template-body file://infra.yaml \
  --capabilities CAPABILITY_IAM
\end{verbatim}

\subsection{Iteration 2 – SSM Agent}
\begin{verbatim}
sudo yum install -y amazon-ssm-agent
sudo systemctl start amazon-ssm-agent
aws ssm start-session --target i-xxxxxxxxxxxx
\end{verbatim}

\subsection{Iteration 3 – Security Hardening}
\begin{verbatim}
aws ec2 revoke-security-group-ingress \
  --group-id sg-0a1b2c3d4e5f6g7h8 \
  --protocol tcp --port 22 --cidr 0.0.0.0/0
\end{verbatim}

\subsection{Iteration 4 – Automated Verification with CodeBuild}
\begin{verbatim}
version: 0.2
phases:
  build:
    commands:
      - echo "Verifying infrastructure..."
      - terraform plan
\end{verbatim}

\section{Resources Used}
\begin{itemize}
  \item S3 Bucket: \texttt{alexbrix-terraform-state}
  \item DynamoDB Table: \texttt{terraform-lock-alexbrix}
  \item EC2 Instance:
    \begin{itemize}
      \item Public IP: \texttt{13.53.124.87}
      \item Private IP: \texttt{10.0.1.15}
      \item Region: \texttt{eu-north-1}
      \item Tag: \texttt{webserver-alexbrix}
      \item Security Group: \texttt{sg-0a1b2c3d4e5f6g7h8}
    \end{itemize}
\end{itemize}

\end{document}
l
=======

\documentclass{article}
\usepackage[utf8]{inputenc}
\usepackage{geometry}
\geometry{margin=1in}
\title{DevOps Infrastructure Documentation}
\author{Alex Brix}
\date{September 09, 2025}

\begin{document}

\maketitle

\section{Level 1 – Web Console}
\begin{itemize}
  \item Login to AWS Web Console
  \item Launch EC2 instance using public AMI
  \item SSH access
  \item Manual configuration
  \item Manual deployment from Git
\end{itemize}

\section{Level 2 – AWS CLI}
\begin{itemize}
  \item Use \texttt{aws ec2 run-instances}
  \item Transition from GUI to CLI
  \item Faster and repeatable flow
\end{itemize}

\section{Level 3 – Infrastructure as Code}

\subsection{Iteration 1 – Basic Terraform}
\begin{verbatim}
terraform init
terraform apply -auto-approve
\end{verbatim}

\subsection{Iteration 2 – Configuration with user-data}
\begin{verbatim}
#!/bin/bash
sudo yum update -y
sudo yum install httpd -y
systemctl start httpd
systemctl enable httpd
\end{verbatim}

\subsection{Iteration 3 – Auto-deploy with user-data}
Destroy everything $
ightarrow$ Add auto-deploy $
ightarrow$ Reconstruct $
ightarrow$ Verify

\subsection{Iteration 4 – Ansible Integration}
\begin{verbatim}
ansible-playbook -i inventory webserver.yml
\end{verbatim}

\subsection{Iteration 5 – Git-based Source}
\begin{verbatim}
git clone https://github.com/alexbrix/infra.git
cd infra
terraform apply -auto-approve
\end{verbatim}

\subsection{Iteration 6 – Shared Backend}
\begin{verbatim}
terraform {
  backend "s3" {
    bucket         = "alexbrix-terraform-state"
    key            = "infra/terraform.tfstate"
    region         = "eu-north-1"
    dynamodb_table = "terraform-lock-alexbrix"
  }
}
\end{verbatim}

\subsection{Iteration 7 – Optimization}
\begin{verbatim}
terraform plan > plan.log
grep "No changes" plan.log && echo "Everything is fine"
\end{verbatim}

\subsection{Iteration 8 – Verification Script}
\begin{verbatim}
#!/bin/bash
terraform plan > verify.log
cat verify.log | grep "No changes"
\end{verbatim}

\section{Level 4 – CI/CD and Security}

\subsection{Iteration 1 – CodePipeline + CloudFormation}
\begin{verbatim}
aws cloudformation create-stack \
  --stack-name InfraStack \
  --template-body file://infra.yaml \
  --capabilities CAPABILITY_IAM
\end{verbatim}

\subsection{Iteration 2 – SSM Agent}
\begin{verbatim}
sudo yum install -y amazon-ssm-agent
sudo systemctl start amazon-ssm-agent
aws ssm start-session --target i-xxxxxxxxxxxx
\end{verbatim}

\subsection{Iteration 3 – Security Hardening}
\begin{verbatim}
aws ec2 revoke-security-group-ingress \
  --group-id sg-0a1b2c3d4e5f6g7h8 \
  --protocol tcp --port 22 --cidr 0.0.0.0/0
\end{verbatim}

\subsection{Iteration 4 – Automated Verification with CodeBuild}
\begin{verbatim}
version: 0.2
phases:
  build:
    commands:
      - echo "Verifying infrastructure..."
      - terraform plan
\end{verbatim}

\section{Resources Used}
\begin{itemize}
  \item S3 Bucket: \texttt{alexbrix-terraform-state}
  \item DynamoDB Table: \texttt{terraform-lock-alexbrix}
  \item EC2 Instance:
    \begin{itemize}
      \item Public IP: \texttt{13.53.124.87}
      \item Private IP: \texttt{10.0.1.15}
      \item Region: \texttt{eu-north-1}
      \item Tag: \texttt{webserver-alexbrix}
      \item Security Group: \texttt{sg-0a1b2c3d4e5f6g7h8}
    \end{itemize}
\end{itemize}

\section{Reflection}
This iteration marks the transition from manual verification to automated workflows. I learned to use CodeBuild effectively, write and test buildspec.yml files, and apply secure flows using CloudFormation ChangeSets. This step improved both the security and efficiency of the infrastructure.

\end{document}
 4680d8d93c464aea93f8f889ef22fed7378cd770
